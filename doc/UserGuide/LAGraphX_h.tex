\begin{verbatim}
int LAGraph_Random_Init
(
    char *msg
) ;
\end{verbatim}




\begin{verbatim}
int LAGraph_Random_Finalize
(
    char *msg
) ;
\end{verbatim}




\begin{verbatim}
int LAGraph_Random_Seed     // construct a random seed vector
(
    // input/output
    GrB_Vector Seed,    // vector of random number seeds, normally GrB_UINT64
    // input
    uint64_t seed,      // scalar input seed
    char *msg
) ;
\end{verbatim}




\begin{verbatim}
int LAGraph_Random_Next     // advance to next random vector
(
    // input/output
    GrB_Vector Seed,
    char *msg
) ;
\end{verbatim}




\begin{verbatim}
GrB_Info LAGraph_Random_Matrix    // random matrix of any built-in type
(
    // output
    GrB_Matrix *A,      // A is constructed on output
    // input
    GrB_Type type,      // type of matrix to construct
    GrB_Index nrows,    // # of rows of A
    GrB_Index ncols,    // # of columns of A
    double density,     // density: build a sparse matrix with
                        // density*nrows*cols values if not INFINITY;
                        // build a dense matrix if INFINITY.
    uint64_t seed,      // random number seed
    char *msg
) ;
\end{verbatim}




\begin{verbatim}
int LAGraph_SWrite_HeaderStart  // write the first part of the JSON header
(
    FILE *f,                    // file to write to
    const char *name,           // name of this collection of matrices
    char *msg
) ;
\end{verbatim}




\begin{verbatim}
int LAGraph_SWrite_HeaderItem   // write a single item to the JSON header
(
    // inputs:
    FILE *f,                    // file to write to
    LAGraph_Contents_kind kind, // matrix, vector, or text
    const char *name,           // name of the matrix/vector/text; matrices from
                                // sparse.tamu.edu use the form "Group/Name"
    const char *type,           // name of type of the matrix/vector
    int compression,            // text compression method
    GrB_Index blob_size,        // exact size of serialized blob for this item
    char *msg
) ;
\end{verbatim}




\begin{verbatim}
int LAGraph_SWrite_HeaderItem   // write a single item to the JSON header
(
    // inputs:
    FILE *f,                    // file to write to
    LAGraph_Contents_kind kind, // matrix, vector, or text
    const char *name,           // name of the matrix/vector/text; matrices from
                                // sparse.tamu.edu use the form "Group/Name"
    const char *type,           // name of type of the matrix/vector
    // todo: text not yet supported by LAGraph_SWrithe_HeaderItem
    int compression,            // text compression method
    GrB_Index blob_size,        // exact size of serialized blob for this item
    char *msg
) ;
\end{verbatim}




\begin{verbatim}
int LAGraph_SWrite_HeaderEnd    // write the end of the JSON header
(
    FILE *f,                    // file to write to
    char *msg
) ;
\end{verbatim}




\begin{verbatim}
int LAGraph_SWrite_Item  // write the serialized blob of a matrix/vector/text
(
    // input:
    FILE *f,                // file to write to
    const void *blob,       // serialized blob from G*B_Matrix_serialize
    GrB_Index blob_size,    // exact size of the serialized blob
    char *msg
) ;
\end{verbatim}




\begin{verbatim}
int LAGraph_SRead   // read a set of matrices from a *.lagraph file
(
    FILE *f,                        // file to read from
    // output
    char **collection,              // name of collection (allocated string)
    LAGraph_Contents **Contents,    // array contents of contents
    GrB_Index *ncontents,           // # of items in the Contents array
    char *msg
) ;
\end{verbatim}




\begin{verbatim}
void LAGraph_SFreeContents      // free the Contents returned by LAGraph_SRead
(
    // input/output
    LAGraph_Contents **Contents,    // array of size ncontents
    GrB_Index ncontents
) ;
\end{verbatim}




\begin{verbatim}
int LAGraph_SSaveSet            // save a set of matrices from a *.lagraph file
(
    // inputs:
    char *filename,             // name of file to write to
    GrB_Matrix *Set,            // array of GrB_Matrix of size nmatrices
    GrB_Index nmatrices,        // # of matrices to write to *.lagraph file
//  todo: handle vectors and text in LAGraph_SSaveSet
    char *collection,           // name of this collection of matrices
    char *msg
) ;
\end{verbatim}




\begin{verbatim}
void LAGraph_SFreeSet           // free a set of matrices
(
    // input/output
    GrB_Matrix **Set_handle,    // array of GrB_Matrix of size nmatrices
    GrB_Index nmatrices         // # of matrices in the set
) ;
\end{verbatim}




\begin{verbatim}
int LAGraph_AllKTruss   // compute all k-trusses of a graph
(
    // outputs
    GrB_Matrix *Cset,   // size n, output k-truss subgraphs
    int64_t *kmax,      // smallest k where k-truss is empty
    int64_t *ntris,     // size max(n,4), ntris [k] is #triangles in k-truss
    int64_t *nedges,    // size max(n,4), nedges [k] is #edges in k-truss
    int64_t *nstepss,   // size max(n,4), nstepss [k] is #steps for k-truss
    // input
    LAGraph_Graph G,    // input graph
    char *msg
) ;
\end{verbatim}




\begin{verbatim}
int LAGraph_KTruss      // compute the k-truss of a graph
(
    // outputs:
    GrB_Matrix *C,      // output k-truss subgraph, C
    // inputs:
    LAGraph_Graph G,    // input graph
    uint32_t k,         // find the k-truss, where k >= 3
    char *msg
) ;
\end{verbatim}




\begin{verbatim}
int LAGraph_cc_lacc (
    GrB_Vector *result,
    GrB_Matrix A,
    bool sanitize,
    char *msg
) ;
\end{verbatim}




\begin{verbatim}
GrB_Info LAGraph_BF_basic
(
    GrB_Vector *pd_output,
    const GrB_Matrix A,
    const GrB_Index s
) ;
\end{verbatim}




\begin{verbatim}
GrB_Info LAGraph_BF_basic_pushpull
(
    GrB_Vector *pd_output,
    const GrB_Matrix A,
    const GrB_Matrix AT,
    const GrB_Index s
) ;
\end{verbatim}




\begin{verbatim}
GrB_Info LAGraph_BF_basic_mxv
(
    GrB_Vector *pd_output,      //the pointer to the vector of distance
    const GrB_Matrix AT,        //transposed adjacency matrix for the graph
    const GrB_Index s           //given index of the source
) ;
\end{verbatim}




\begin{verbatim}
GrB_Info LAGraph_BF_full
(
    GrB_Vector *pd_output,
    GrB_Vector *ppi_output,
    GrB_Vector *ph_output,
    const GrB_Matrix A,
    const GrB_Index s
) ;
\end{verbatim}




\begin{verbatim}
GrB_Info LAGraph_BF_full1
(
    GrB_Vector *pd_output,
    GrB_Vector *ppi_output,
    GrB_Vector *ph_output,
    const GrB_Matrix A,
    const GrB_Index s
) ;
\end{verbatim}




\begin{verbatim}
GrB_Info LAGraph_BF_full1a
(
    GrB_Vector *pd_output,
    GrB_Vector *ppi_output,
    GrB_Vector *ph_output,
    const GrB_Matrix A,
    const GrB_Index s
) ;
\end{verbatim}




\begin{verbatim}
GrB_Info LAGraph_BF_full2
(
    GrB_Vector *pd_output,      //the pointer to the vector of distance
    GrB_Vector *ppi_output,     //the pointer to the vector of parent
    GrB_Vector *ph_output,      //the pointer to the vector of hops
    const GrB_Matrix A,         //matrix for the graph
    const GrB_Index s           //given index of the source
) ;
\end{verbatim}




\begin{verbatim}
GrB_Info LAGraph_BF_full_mxv
(
    GrB_Vector *pd_output,
    GrB_Vector *ppi_output,
    GrB_Vector *ph_output,
    const GrB_Matrix AT,
    const GrB_Index s
) ;
\end{verbatim}




\begin{verbatim}
GrB_Info LAGraph_BF_pure_c
(
    int32_t **pd,

    int64_t **ppi,

    const int64_t s,
    const int64_t n,
    const int64_t nz,
    const int64_t *I,
    const int64_t *J,
    const int32_t *W
) ;
\end{verbatim}




\begin{verbatim}
GrB_Info LAGraph_BF_pure_c_double
(
    double **pd,

    int64_t **ppi,

    const int64_t s,
    const int64_t n,
    const int64_t nz,
    const int64_t *I,
    const int64_t *J,
    const double  *W
) ;
\end{verbatim}




\begin{verbatim}
int LAGraph_cdlp
(
    GrB_Vector *CDLP_handle,
    const GrB_Matrix A,
    bool symmetric,
    bool sanitize,
    int itermax,
    double *t,
    char *msg
) ;
\end{verbatim}




\begin{verbatim}
GrB_Info LAGraph_dnn
(
    // output
    GrB_Matrix *Yhandle,
    // input: not modified
    GrB_Matrix *W,
    GrB_Matrix *Bias,
    int nlayers,
    GrB_Matrix Y0
) ;
\end{verbatim}




\begin{verbatim}
GrB_Info LAGraph_FW
(
    const GrB_Matrix G,
    GrB_Matrix *D,
    GrB_Type   *D_type
) ;
\end{verbatim}




\begin{verbatim}
int LAGraph_lcc            // compute lcc for all nodes in A
(
    GrB_Vector *LCC_handle,     // output vector
    const GrB_Matrix A,         // input matrix
    bool symmetric,             // if true, the matrix is symmetric
    bool sanitize,              // if true, ensure A is binary
    double t [2],               // t [0] = sanitize time, t [1] = lcc time,
                                // in seconds
    char *msg
) ;
\end{verbatim}




\begin{verbatim}
int LAGraph_msf
(
    GrB_Matrix *result, // output: an unsymmetrical matrix, the spanning forest
    GrB_Matrix A,       // input matrix
    bool sanitize,      // if true, ensure A is symmetric
    char *msg
) ;
\end{verbatim}




\begin{verbatim}
int LAGraph_scc (
    GrB_Vector *result,     // output: array of component identifiers
    GrB_Matrix A,           // input matrix
    char *msg
) ;
\end{verbatim}




\begin{verbatim}
int LAGraph_VertexCentrality_Triangle       // vertex triangle-centrality
(
    // outputs:
    GrB_Vector *centrality,     // centrality(i): triangle centrality of i
    uint64_t *ntriangles,       // # of triangles in the graph
    // inputs:
    int method,                 // 0, 1, 2, or 3
    LAGraph_Graph G,            // input graph
    char *msg
) ;
\end{verbatim}




\begin{verbatim}
int LAGraph_MaximalIndependentSet       // maximal independent set
(
    // outputs:
    GrB_Vector *mis,            // mis(i) = true if i is in the set
    // inputs:
    LAGraph_Graph G,            // input graph
    uint64_t seed,              // random number seed
    GrB_Vector ignore_node,     // if NULL, no nodes are ignored.  Otherwise
                                // ignore_node(i) = true if node i is to be
                                // ignored, and not treated as a candidate
                                // added to maximal independent set.
    char *msg
) ;
\end{verbatim}




\begin{verbatim}
int LG_CC_FastSV5           // SuiteSparse:GraphBLAS method, with GxB extensions
(
    // output
    GrB_Vector *component,  // output: array of component identifiers
    // inputs
    LAGraph_Graph G,        // input graph, modified then restored
    char *msg
) ;
\end{verbatim}




\begin{verbatim}
int LAGraph_KCore_All
(
    // outputs:
    GrB_Vector *decomp,     // kcore decomposition
    uint64_t *kmax,
    // inputs:
    LAGraph_Graph G,            // input graph
    char *msg
) ;
\end{verbatim}




\begin{verbatim}
int LAGraph_KCore
(
    // outputs:
    GrB_Vector *decomp,     // kcore decomposition
    // inputs:
    LAGraph_Graph G,        // input graph
    uint64_t k,             //k level to compare to
    char *msg
) ;
\end{verbatim}




\begin{verbatim}
int LAGraph_KCore_Decompose
(
    // outputs:
    GrB_Matrix *D,              // kcore decomposition
    // inputs:
    LAGraph_Graph G,            // input graph
    GrB_Vector decomp,         // input decomposition matrix
    uint64_t k,
    char *msg
) ;
\end{verbatim}




\begin{verbatim}
int LAGraph_FastGraphletTransform
(
    // outputs:
    GrB_Matrix *F_net,  // 16-by-n matrix of graphlet counts
    // inputs:
    LAGraph_Graph G,
    bool compute_d_15,  // probably this makes most sense
    char *msg
) ;
\end{verbatim}




\begin{verbatim}
int LAGraph_SquareClustering
(
    // outputs:
    GrB_Vector *square_clustering,
    // inputs:
    LAGraph_Graph G,
    char *msg
) ;
\end{verbatim}




\begin{verbatim}
typedef enum
{
    LAGraph_unknown_kind = -1,  // unknown kind
    LAGraph_matrix_kind = 0,    // a serialized GrB_Matrix
    LAGraph_vector_kind = 1,    // a serialized GrB_Vector (SS:GrB only)
    LAGraph_text_kind = 2,      // text (char *), possibly compressed
}
LAGraph_Contents_kind ;
\end{verbatim}




