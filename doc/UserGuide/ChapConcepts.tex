\chapter{Basic concepts}
\label{Chp:Concepts}

The LAGraph library is a collection of high level graph algorithms
based on the GraphBLAS C API.  These algorithms construct  
graph algorithms expressed ``in the language of linear algebra.''
Graphs are expressed as matrices, and the operations over 
these matrices are generalized through the use of a
semiring algebraic structure.

In this chapter, we will define the basic concepts used to
define the LAGraph Library  We provide the following elements:

\begin{itemize}
\item Glossary of terms and notation used in this document.  
\item The LAGraph objects. 
\item Return codes and other constants used in LAGraph.
\end{itemize}

Currently, I've kept the text from the GraphBLAS concepts chapter in 
this document.  We may want to borrow some of the GraphBLAS glossary items
and perhaps use some of the table formatting in LAGraph.

\section{Glossary}

%=============================================================================

\subsection{LAGraph basic definitions}

\glossBegin

\glossItem{application} A program that calls methods from the GraphBLAS C API to
solve a problem.

\glossItem{GraphBLAS C API} The application programming interface that fully defines the types, objects, 
literals, and other elements of the C binding to the GraphBLAS.

\glossItem{function} Refers to a named group of statements in the C programming language.  Methods, operators,
and user-defined functions are typically implemented as C functions.
When referring to 
the code programmers write, as opposed to the role of functions as an element of the GraphBLAS, they may
be referred to as such.

\glossItem{method} A function defined in the GraphBLAS C API that manipulates
GraphBLAS objects or other opaque features of the implementation of the GraphBLAS API.

\glossItem{operator} A function that performs an operation on the elements stored in GraphBLAS matrices and vectors.

\glossItem{GraphBLAS operation} A mathematical operation defined in the
GraphBLAS mathematical specification. These operations (not to be confused with \emph{operators}) typically act
on matrices and vectors with elements defined in terms of an algebraic semiring. 
\glossEnd

%=============================================================================

\subsection{LAGraph objects and their structure}

\glossBegin
\glossItem{non-opaque datatype} Any datatype that exposes its internal structure and
can be manipulated directly by the user.   

\glossItem{opaque datatype} Any datatype that hides its internal structure and can
be manipulated only through an API.

\glossItem{GraphBLAS object}  An instance of an \emph{opaque datatype} defined 
by the \emph{GraphBLAS C API} that is manipulated only through the GraphBLAS 
API. There are four kinds of GraphBLAS opaque objects: \emph{domains} (i.e., types), 
\emph{algebraic objects} (operators, monoids and semirings), 
\emph{collections} (scalars, vectors, matrices and masks), and descriptors.   

\glossItem{handle}  A variable that holds a reference to an instance of one of 
the GraphBLAS opaque objects.  The value of this variable holds a reference to 
a GraphBLAS object but not the contents of the object itself.  Hence, assigning 
a value to another variable copies the reference to the GraphBLAS object of one 
handle but not the contents of the object.

\glossItem{domain} The set of valid values for the elements stored in a 
GraphBLAS \emph{collection} or operated on by a GraphBLAS \emph{operator}.
Note that some GraphBLAS objects involve functions that map values from 
one or more input domains onto values in an output domain.  These GraphBLAS 
objects would have multiple domains.

\glossItem{collection} An opaque GraphBLAS object that holds a number of
elements from a specified \emph{domain}. Because these objects are based on an 
opaque datatype, an implementation of the GraphBLAS C API has the flexibility 
to optimize the data structures for a particular platform.  GraphBLAS objects 
are often implemented as sparse data structures, meaning only the subset of the
elements that have values are stored.

\glossItem{implied zero}  Any element that has a valid index (or indices) 
in a GraphBLAS vector or matrix but is not explicitly identified in the list of 
elements of that vector or matrix. From a mathematical perspective, an
\emph{implied zero} is treated as having the 
value of the zero element of the relevant monoid or semiring.
However, GraphBLAS operations are purposefully defined using set notation in such a way
that it makes it unnecessary to reason about implied zeros. 
Therefore, this concept is not used in the definition of GraphBLAS methods and operators.

\glossItem{mask} An internal GraphBLAS object used to control how values 
are stored in a method's output object.  The mask exists only inside a method; hence,
it is called an \emph{internal opaque object}.  A mask is formed from the elements of
a collection object (vector or matrix) input as a mask parameter to a method. GraphBLAS 
allows two types of masks:
\begin{enumerate}
\item In the default 
case, an element of the mask exists for each element that exists in the 
input collection object when the value of that element, when cast to a Boolean type, evaluates to 
{\tt true}.  
\item In the {\it structure only} case, masks have structure but no values. 
The input collection describes a structure whereby an 
element of the mask exists for each element stored in the input collection regardless of its value.
\end{enumerate}

\glossItem{complement} The \emph{complement} of a 
GraphBLAS mask, $M$, is another mask, $M'$, where the elements of $M'$
are those elements from $M$ that \emph{do not} exist.  
\glossEnd

%=============================================================================

\subsection{Algebraic structures used in the GraphBLAS}

\glossBegin
\glossItem{associative operator} In an expression where a binary operator is used 
two or more times consecutively, that operator is \emph{associative} if the result 
does not change regardless of the way operations are grouped (without changing their order). 
In other words, in a sequence of binary operations using the same associative 
operator, the legal placement of parenthesis does not change the value resulting 
from the sequence operations.  Operators that are associative over infinitely 
precise numbers (e.g., real numbers) are not strictly associative when applied to 
numbers with finite precision (e.g., floating point numbers). Such non-associativity 
results, for example, from roundoff errors or from the fact some numbers can not 
be represented exactly as floating point numbers.   In the GraphBLAS specification, 
as is common practice in computing, we refer to operators as \emph{associative} 
when their mathematical definition over infinitely precise numbers is associative 
even when they are only approximately associative when applied to finite precision 
numbers.

No GraphBLAS method will imply a predefined grouping over any associative operators. 
Implementations of the GraphBLAS are encouraged to exploit associativity to optimize 
performance of any GraphBLAS method with this requirement. This holds even if the 
definition of the GraphBLAS method implies a fixed order for the associative operations.

\glossItem{commutative operator} In an expression where a binary operator is used (usually
two or more times consecutively), that operator is \emph{commutative} if the result does 
not change regardless of the order the inputs are operated on.

No GraphBLAS method will imply a predefined ordering over any commutative operators. 
Implementations of the GraphBLAS are encouraged to exploit commutativity to optimize 
performance of any GraphBLAS method with this requirement. This holds even if the 
definition of the GraphBLAS method implies a fixed order for the commutative operations.

\glossItem{GraphBLAS operators} Binary or unary operators that act on elements of GraphBLAS 
objects.  \emph{GraphBLAS operators} are used to express algebraic structures used in the 
GraphBLAS such as monoids and semirings. They are also used as arguments to several
GraphBLAS methods. There are two types of \emph{GraphBLAS operators}: 
(1) predefined operators found in Table~\ref{Tab:PredefOperators} and (2) user-defined 
operators created using {\sf GrB\_UnaryOp\_new()} or {\sf GrB\_BinaryOp\_new()}.

\glossItem{monoid} An algebraic structure consisting of one domain, an associative 
binary operator, and the identity of that operator.  There are two types 
of GraphBLAS monoids: (1) predefined monoids found in 
Table~\ref{Tab:PredefinedMonoids} and (2) user-defined monoids created using . 

\glossItem{semiring} An algebraic structure consisting of a set of allowed values
(the \emph{domain}), a commutative and associative binary operator called addition, a binary operator 
called multiplication (where multiplication distributes over addition),
and identities over addition (\emph{0}) and multiplication (\emph{1}).  The additive
identity is an annihilator over multiplication.   

\glossItem{GraphBLAS semiring} is allowed to diverge from the mathematically 
rigorous definition of a \emph{semiring} since certain combinations of domains, operators, and identity 
elements are useful in graph algorithms even when they do not strictly match the mathematical
definition of a semiring.
There are two types 
of \emph{GraphBLAS semirings}: (1) predefined semirings found in 
Tables~\ref{Tab:PredefinedTrueSemirings} and~\ref{Tab:PredefinedUsefulSemirings}, and (2) user-defined semirings created using 
{\sf GrB\_Semiring\_new()} (see Section~\ref{Sec:AlgebraMethods}).

\glossItem{index unary operator} A variation of the unary operator that operates
on elements of GraphBLAS vectors and matrices along with the index values 
representing their location in the objects.  There are predefined index unary
operators found in Table~\ref{Tab:PredefIndexOperators}), and user-defined
operators created using {\sf GrB\_IndexUnaryOp\_new} (see Section~\ref{Sec:AlgebraMethods}).
\glossEnd

%=============================================================================

\subsection{The execution of an application using the GraphBLAS C API}

\glossBegin
\glossItem{program order} The order of the GraphBLAS method calls in a
thread, as defined by the text of the program.

\glossItem{host programming environment} The GraphBLAS specification defines an API.  
The functions from the API appear in a program.  This program is written using a programming language
and execution environment defined outside of the GraphBLAS.  We refer to this programming environment
as the ``host programming environment''.

\glossItem{execution time} time expended while executing instructions defined by a program.
This term is specifically used in this specification in the context of computations 
carried out on behalf of a call to a GraphBLAS method.

%% The original definition was too narrow by only referencing GraphBLAS methods.
%% Also, I didn't see the reason why the comment about execution time was retrieved was included.
%\glossItem{execution time} The time it takes to execute a GraphBLAS
%method call. Implementations are free, but not mandated, to specify how
%the execution time of a method call can be retrieved.

\glossItem{sequence} A GraphBLAS application uniquely defines a directed
acyclic graph (DAG) of GraphBLAS method calls based on their program order.  
At any point in a program, the state of any GraphBLAS object is defined by a 
subgraph of that DAG.  An ordered collection of GraphBLAS method calls in program order that
defines that subgraph for a particular object is the \emph{sequence} for that object.

\glossItem{complete}  A GraphBLAS object is complete when it can be used in a happens-before relationship
with a method call that reads the variable on another thread.  This concept is used
when reasoning about memory orders in multithreaded programs.  A GraphBLAS object defined on one thread 
that is complete can be safely used as an {\sf IN} or {\sf INOUT} argument
in a method-call on a second thread assuming the method calls are correctly synchronized so the definition on 
the first thread \emph{happens-before} it is used on the second thread.  In blocking-mode, an object is 
complete after a GraphBLAS method call that writes to that object returns.   In nonblocking-mode, an object is complete 
after a call to the {\sf GrB\_wait()} method with the {\sf GrB\_COMPLETE} parameter.

\glossItem{materialize} A GraphBLAS object is materialized when it is (1) complete, (2) the computations 
defined by the sequence that define the object have finished (either fully or stopped at an error) and will not consume any 
additional computational resources, and (3) any errors associated with that sequence are available to be read according to the 
GraphBLAS error model.  A GraphBLAS object that is never loaded into a non-opaque data structure may 
potentially never be materialized.  This might happen, for example, if the operations 
associated with the object are fused or otherwise changed by the runtime system 
that supports the implementation of the GraphBLAS C API.   An object can be materialized by a call
to the materialize mode of the {\sf GrB\_wait()} method. 

\glossItem{context}  An instance of the GraphBLAS C API implementation
as seen by an application.  An application can have only one context between the 
start and end of the application.  
A context begins with the first thread that calls {\sf GrB\_init()} and ends with the 
first thread to call {\sf GrB\_finalize()}.  
It is an error for {\sf GrB\_init()} or {\sf GrB\_finalize()} to be called more than one
time within an application.  The context is used to constrain the behavior of an
instance of the GraphBLAS C API implementation and support various execution strategies.
Currently, the only
supported constraints on a context pertain to the mode of program execution.

\glossItem{program execution mode} Defines how a GraphBLAS sequence executes, and is associated 
with the {\it context} of a GraphBLAS C API implementation. It is set by an 
application with its call to {\sf GrB\_init()} to one of two possible states.  
In \emph{blocking mode}, GraphBLAS methods return after the computations 
complete and any output objects have been materialized.  In {\it nonblocking mode}, a 
method may return once the arguments are tested as consistent with 
the method (\ie, there are no API errors), and potentially before any computation 
has taken place.
\glossEnd

%=============================================================================

\subsection{GraphBLAS methods: behaviors and error conditions}
\glossBegin
\glossItem{implementation-defined behavior} Behavior that must be documented
by the implementation and is allowed to vary among different
compliant implementations. 

\glossItem{undefined behavior} Behavior that is not specified by the GraphBLAS C API.
A conforming implementation is free to choose results delivered from a method
whose behavior is undefined. 

\glossItem{thread-safe}  Consider a function called from multiple threads with 
arguments that do not overlap in memory (i.e. the argument lists do not share 
memory).  If the function is \emph{thread-safe} then it will behave the same 
when executed concurrently by multiple threads or sequentially on a single 
thread.

\glossItem{dimension compatible} GraphBLAS objects (matrices and vectors) that are
passed as parameters to a GraphBLAS method are dimension (or shape) compatible if
they have the correct number of dimensions and sizes for each dimension to satisfy 
the rules of the mathematical definition of the operation associated with the method. 
If any \emph{dimension compatibility} rule above is violated, execution of the GraphBLAS 
method ends and the {\sf GrB\_DIMENSION\_MISMATCH} error is returned.

\glossItem{domain compatible} Two domains for which values from one domain can be 
cast to values in the other domain as per the rules of the C language. In particular, 
domains from Table~\ref{Tab:PredefinedTypes} 
are all compatible with each other, and a domain from a user-defined type is only 
compatible with itself. If any \emph{domain compatibility} rule above is 
violated, execution of the GraphBLAS method ends and the {\sf GrB\_DOMAIN\_MISMATCH} 
error is returned.
\glossEnd

\vfill

\newgeometry{left=2.5cm,top=2cm,bottom=2cm}

%=============================================================================
%=============================================================================

\section{Notation}

\begin{tabular}[H]{l|p{5in}}
Notation & Description \\
\hline
$\Dout, \Dinn, \Din1, \Din2$  & Refers to output and input domains of various GraphBLAS operators. \\
$\bDout(*), \bDinn(*),$ & Evaluates to output and input domains of GraphBLAS operators (usually \\
~~~~$\bDin1(*), \bDin2(*)$ & a unary or binary operator, or semiring). \\
$\mathbf{D}(*)$   & Evaluates to the (only) domain of a GraphBLAS object (usually a monoid, vector, or matrix). \\ 
$f$             & An arbitrary unary function, usually a component of a unary operator. \\
$\mathbf{f}(F_u)$ & Evaluates to the unary function contained in the unary operator given as the argument. \\
$\odot$         & An arbitrary binary function, usually a component of a binary operator. \\
$\mathbf{\bigodot}(*)$ & Evaluates to the binary function contained in the binary operator or monoid given as the argument. \\
$\otimes$       & Multiplicative binary operator of a semiring. \\
$\oplus$        & Additive binary operator of a semiring. \\
$\mathbf{\bigotimes}(S)$ & Evaluates to the multiplicative binary operator of the semiring given as the argument. \\
$\mathbf{\bigoplus}(S)$ & Evaluates to the additive binary operator of the semiring given as the argument. \\
$\mathbf{0}(*)$   & The identity of a monoid, or the additive identity of a GraphBLAS semiring. \\
$\mathbf{L}(*)$   & The contents (all stored values) of the vector or matrix GraphBLAS objects.  For a vector, it is the set of (index, value) pairs, and for a matrix it is the set of (row, col, value) triples. \\
$\mathbf{v}(i)$ or $v_i$   & The $i^{th}$ element of the vector $\vector{v}$.\\
$\mathbf{size}(\vector{v})$ & The size of the vector $\vector{v}$.\\
$\mathbf{ind}(\vector{v})$ & The set of indices corresponding to the stored values of the vector $\vector{v}$.\\
$\mathbf{nrows}(\vector{A})$ & The number of rows in the $\matrix{A}$.\\
$\mathbf{ncols}(\vector{A})$ & The number of columns in the $\matrix{A}$.\\
$\mathbf{indrow}(\vector{A})$ & The set of row indices corresponding to rows in $\matrix{A}$ that have stored values.  \\
$\mathbf{indcol}(\vector{A})$ & The set of column indices corresponding to columns in $\matrix{A}$ that have stored values. \\
$\mathbf{ind}(\vector{A})$ & The set of $(i,j)$ indices corresponding to the stored values of the matrix. \\
$\mathbf{A}(i,j)$ or $A_{ij}$ & The element of $\matrix{A}$ with row index $i$ and column index $j$.\\
$\matrix{A}(:,j)$ & The $j^{th}$ column of matrix $\matrix{A}$.\\
$\matrix{A}(i,:)$ & The $i^{th}$ row of matrix $\matrix{A}$.\\
$\matrix{A}^T$ &The transpose of matrix $\matrix{A}$. \\
$\neg\matrix{M}$ & The complement of $\matrix{M}$.\\
s$(\matrix{M})$ & The structure of $\matrix{M}$.\\
$\vector{\widetilde{t}}$ & A temporary object created  by the GraphBLAS implementation. \\
$<type>$ & A method argument type that is {\sf void *} or one of the types from Table~\ref{Tab:PredefinedTypes}. \\
{\sf GrB\_ALL} & A method argument literal to indicate that all indices of an input array should be used.\\
{\sf GrB\_Type} & A method argument type that is either a user defined type or one of the  types from Table~\ref{Tab:PredefinedTypes}.\\
{\sf GrB\_Object} &  A method argument type referencing any of the GraphBLAS object types.\\
{\sf GrB\_NULL} & The GraphBLAS NULL.
\end{tabular}

\restoregeometry


\section{Mathematical foundations}

Graphs can be represented in terms of matrices. The values stored in these 
matrices correspond to attributes (often weights) of edges in the graph.\footnote{More information on the mathematical foundations can be found in the following paper: J. Kepner, P. Aaltonen, D. Bader,  A. Buluç, F. Franchetti, J. Gilbert, D. Hutchison, M. Kumar, A. Lumsdaine, H. Meyerhenke, S. McMillan, J. Moreira, J. Owens, C. Yang, M. Zalewski, and T. Mattson. 2016, September. Mathematical foundations of the GraphBLAS. In \emph{2016 IEEE High Performance Extreme Computing Conference (HPEC)} (pp. 1-9). IEEE.} 
Likewise, information about vertices in a graph are stored in vectors.
The set of valid values that can be stored in either matrices or vectors
is referred to as their domain. Matrices are usually sparse because the 
lack of an edge between two vertices means that nothing is stored at the 
corresponding location in the matrix.  Vectors may be sparse or dense, or they may 
start out sparse and become dense as algorithms traverse the graphs.

Operations defined by the GraphBLAS C API specification operate on these 
matrices and vectors to carry out graph algorithms.  These GraphBLAS 
operations are defined in terms of GraphBLAS semiring algebraic 
structures. Modifying the underlying semiring changes the result of 
an operation to support a wide range of graph algorithms.
Inside a given algorithm, it is often beneficial to change the GraphBLAS 
semiring that applies to an operation on a matrix.  This has two 
implications for the C binding of the GraphBLAS API.  

First, it means that we define a separate object for the semiring 
to pass into methods.  Since in many cases the full
semiring is not required, we also support passing monoids or
even binary operators, which means the semiring is implied rather than 
explicitly stated.

Second, the ability to change semirings impacts the meaning of 
the \emph{implied zero} in a sparse representation of a matrix or vector.
This element in real arithmetic is zero, which is the 
identity of the \emph{addition} operator and the annihilator of the
\emph{multiplication} operator.  As the semiring changes, this 
implied zero changes to the identity of the \emph{addition} operator 
and the annihilator (if present) of the \emph{multiplication} operator 
for the new semiring. Nothing changes regarding what is stored in the sparse 
matrix or vector, but the implied zeros within them change with respect to a 
particular operation. In all cases, the nature of the implied zero does not 
matter since the GraphBLAS C API requires that implementations treat them as 
nonexistent elements of the matrix or vector.

As with matrices and vectors, GraphBLAS semirings have domains
associated with their inputs and outputs.  The semirings in the 
GraphBLAS C API are defined with two domains associated with the input operands and one 
domain associated with output.  When used in the GraphBLAS C API these
domains may not match the domains of the matrices and vectors supplied in
the operations.  In this case, only valid \emph{domain compatible} casting 
is supported by the API.

The mathematical formalism for graph operations in the language of 
linear algebra often assumes that we can operate in the field of real numbers. 
However, the GraphBLAS C binding is designed for implementation on computers, 
which by necessity have a finite number of bits to represent numbers. 
Therefore, we require a conforming implementation to use floating point 
numbers such as those defined by the IEEE-754 standard (both single- and double-precision) 
wherever real numbers need to be represented. The practical implications of 
these finite precision numbers is that the result of a sequence of 
computations may vary from one execution to the next as the grouping of operands
(because of associativity) within the operations changes.  While techniques are known to 
reduce these effects, we do not require or even expect an implementation 
to use them as they may add considerable overhead. In most 
cases, these roundoff errors are not significant. When they are significant, 
the problem itself is ill-conditioned and needs to be reformulated.


\section{LAgraph  objects}

Objects defined in the GraphBLAS standard include types (the domains of 
elements), collections of elements (matrices, vectors, and scalars), operators 
on those elements (unary, index unary, and binary operators), algebraic 
structures (semirings and monoids), and descriptors.   GraphBLAS objects are 
defined as opaque types; that is, they are managed, manipulated, and accessed 
solely through the GraphBLAS application programming interface. This gives an 
implementation of the GraphBLAS C specification flexibility to optimize objects 
for different scenarios or to meet the needs of different hardware platforms.

A GraphBLAS opaque object is accessed through its \emph{handle}.  A handle is 
a variable that references an instance of one of the types from 
Table~\ref{Tab:ObjTypes}.  An implementation of the GraphBLAS specification 
has a great deal of flexibility in how these handles are implemented.  All 
that is required is that the handle corresponds to a type defined in the 
C language that supports assignment and comparison for equality.  The
GraphBLAS specification defines a literal {\sf GrB\_INVALID\_HANDLE} that is 
valid for each type.  Using the logical equality operator from C, it must be 
possible to compare a handle to {\sf GrB\_INVALID\_HANDLE} to verify that a 
handle is valid.


\begin{table}
\hrule
\begin{center}
\caption{Types of GraphBLAS opaque objects.}
\label{Tab:ObjTypes}
~\\
\begin{tabular}{l|l}
{\sf GrB\_Object types} & Description \\
\hline
{\sf GrB\_Type}           & Scalar type.     \\ \hline
{\sf GrB\_UnaryOp}        & Unary operator.     \\
{\sf GrB\_IndexUnaryOp}   & Unary operator, that operates on a single value and its location index values.     \\
{\sf GrB\_BinaryOp}       & Binary operator.     \\
{\sf GrB\_Monoid}         & Monoid algebraic structure.     \\
{\sf GrB\_Semiring}       & A GraphBLAS semiring algebraic structure. \\ \hline
{\sf GrB\_Scalar}         & One element; could be empty. \\ 
{\sf GrB\_Vector}         & One-dimensional collection of elements; can be sparse.     \\
{\sf GrB\_Matrix}         & Two-dimensional collection of elements; typically sparse.    \\ \hline
{\sf GrB\_Descriptor}     & Descriptor object, used to modify behavior of methods (specifically \\
                          & GraphBLAS operations).     \\
\end{tabular}
\end{center}
\hrule
\end{table}

Every GraphBLAS object has a \emph{lifetime}, which consists of
the sequence of instructions executed in program order between the
\emph{creation} and the \emph{destruction} of the object.  The GraphBLAS C
API predefines a number of these objects which are created
when the GraphBLAS context is initialized by a call to {\sf GrB\_init}
and are destroyed when the GraphBLAS context is terminated by a call to
{\sf GrB\_finalize}.

An application using the GraphBLAS API can create additional objects by
declaring variables of the appropriate type from Table~\ref{Tab:ObjTypes} for 
the objects it will use.  Before use, the object must be initialized 
with a call call to one of the object's respective \emph{constructor} methods.  
Each kind of object has at least one explicit constructor method of the form 
{\sf GrB\_*\_new} where `{\sf *}' is replaced with the type of object (e.g., 
{\sf GrB\_Semiring\_new}). Note that some objects, especially collections, 
have additional constructor methods such as duplication, import, or 
deserialization.  Objects explicitly created by a call to a constructor 
should be destroyed by a call to {\sf GrB\_free}. The behavior of a program
that calls {\sf GrB\_free} on a pre-defined object is undefined.

%This is typically done with one of 
%the methods that has a ``{\sf \_new}'' suffix in its name (e.g., 
%{\sf GrB\_Vector\_new}).  If available, an object can also be initialized by 
%duplicating an existing object with one of the methods that has the 
%``{\sf \_dup}'' suffix in its name  (e.g., {\sf GrB\_Vector\_dup}).  Note that 
%there are other valid constructor methods included in the API (e.g., 
%``{\sf \_diag}'', ``{\sf \_import}'', and ``{\sf \_deserialize}'' matrix 
%methods).  Regardless of the method of construction, any resources associated 
%with that object can be released (destructed) by a call to the {\sf GrB\_free} 
%method when an application is finished with an object.    

These constructor and destructor methods are the only methods that change 
the value of a handle.  Hence, objects changed by these methods are passed
into the method as pointers.  In all other cases, handles are not changed by the 
method and are passed by value.  For example, even when multiplying matrices, 
while the contents of the output product matrix changes, the handle for that 
matrix is unchanged. 

Several GraphBLAS constructor methods take other objects as input arguments
and use these objects to create a new object. For all these
methods, the lifetime of the created object must end strictly before
the lifetime of any dependent input objects. For example, a vector constructor
{\sf GrB\_Vector\_new} takes a {\sf GrB\_Type} object as input. That type
object must not be destroyed until after the created vector is destroyed.
Similarly, a {\sf GrB\_Semiring\_new} method takes a monoid and
a binary operator as inputs. Neither of these can be destroyed until
after the created semiring is destroyed.

Note that some constructor methods like {\sf GrB\_Vector\_dup} and 
{\sf GrB\_Matrix\_dup} behave differently. In these cases, the input 
vector or matrix can
be destroyed as soon as the call returns. However, the original type
object used to create the input vector or matrix cannot be destroyed
until after the vector or matrix created by {\sf GrB\_Vector\_dup} or
{\sf GrB\_Matrix\_dup} is destroyed.  This behavior must hold for any
chain of duplicating constructors.

Programmers using GraphBLAS handles must be careful to distinguish between a 
handle and the object manipulated through a handle.  For example, a program may 
declare two GraphBLAS objects of the same type, initialize one, and then assign 
it to the other variable.  That assignment, however, only assigns the handle to 
the variable.  It does not create a copy of that variable (to do that, one 
would need to use the appropriate duplication method).  If later the object is 
freed by calling {\sf GrB\_free} with the first variable, the object is 
destroyed and the second variable is left referencing an object that no longer 
exists (a so-called ``dangling handle'').

In addition to opaque objects manipulated through handles, the GraphBLAS C API 
defines an additional opaque object as an internal object; that is, the object 
is never exposed as a variable within an application.  This opaque object is 
the mask used to control which computed values can be stored in the output 
operand of a \emph{GraphBLAS operation}.  .

